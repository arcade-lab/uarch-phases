\begin{abstract}
Dynamic reconfiguration systems guided by coarse-grained program phases has found success in improving overall program performance and energy efficiency. These performance/energy savings are limited by the granularity that program phases are detected since phases that occur at a finer granularity goes undetected and reconfiguration opportunities are missed.  In this study, we detect program phases using interval sizes on the order of tens, hundreds, and thousands of program cycles. This is in stark contrast with prior phase detection studies where the interval size is on the order of several thousands to millions of cycles.  The primary goal of this study is to begin to fill a gap in the literature on phase detection by characterizing super fine-grained program phases and demonstrating an application where detection of these relatively short-lived phases can be instrumental. Traditional models for phase detection including basic block vectors and working set signatures are used to detect super fine-grained phases as well as a less traditional model based on microprocessor activity. Finally, we show an analytical case study where super fine-grained phases are applied to voltage and frequency scaling optimizations. 
\end{abstract}

